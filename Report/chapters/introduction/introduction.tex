\chapter{Introduction}\label{ch:intro}
Several warehouses around the world are starting to use robots for picking groceries of different sorts \citep{Olsen2018, Perez2018, Vincent2018}. This is done to automate the process of grocery picking and make the entire process faster, to be able to deliver packages faster to the consumer. \cite{Olsen2018, Perez2018, Vincent2018} present robots picking from boxes and not free standing products like in a general supermarket.\\

The goal of Aivero is to enable robots in a general supermarket set-up, to pick groceries straight from the shelves with only the information of which section of a shelf the desired grocery is. The aim is to do this using a video stream directly from the robot and process this data in real time. Using the video stream a processing unit must identify the desired grocery and find the optimal picking point on the item. For a potential higher accuracy the goal is to use both a regular colour video stream, RGB, and depth video. This will enable the robot to see the groceries in physical shape better from the depth video, but also use th RGB video stream to recognise e.g. labels on a product.

\chapter{Aivero AS}
Aivero is small company with nine employees including the intern. The aim of the company is to develop skills for robotics via 3D vision and AI. The first implementation done towards  the goal was to be able to stream and store \gls{rgbd} video data. For the storing to be viable, because of the amount of space \gls{rgbd} video can take up, a compression algorithm is needed, which has been developed and works on PC and is implementable on an Nvidia TX2.\\

The company has offices in Stavanger, Norway and Aalborg, Denmark. Three people in Norway, where the CEO, board director and marketing are located. Four people in Aalborg, with the CTO, two student developers and myself doing full time as an intern. Two other developers are part of the team. One software developer in Aarhus, Denmark, and another part time developer located in Delhi, India. The company was started in December 2017. My responsibility area is AI development and planning, as the only one in the company with a bit more extensive knowledge in deep learning and computer vision.

